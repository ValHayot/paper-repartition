\documentclass[conference]{IEEEtran}
\IEEEoverridecommandlockouts
% The preceding line is only needed to identify funding in the first footnote. If that is unneeded, please comment it out.
\usepackage{cite}
\usepackage{amsmath,amssymb,amsfonts}
\usepackage{graphicx}
\usepackage{textcomp}
\usepackage{xcolor}
\usepackage{algorithmic, algorithm}
\def\BibTeX{{\rm B\kern-.05em{\sc i\kern-.025em b}\kern-.08em
    T\kern-.1667em\lower.7ex\hbox{E}\kern-.125emX}}

\newcommand{\logic}[1]{\color{red}\textbf{Logic/flow:}#1\color{black}}
\newcommand{\writing}[1]{\color{green}\textbf{Writing:}#1\color{black}}
\newcommand{\tristan}[1]{\color{orange}\textbf{From Tristan:}#1\color{black}}

\begin{document}

\title{ Dask IO for sequentially splitting, merging and resplitting multidimensional arrays }

\author{\IEEEauthorblockN{Timoth\'ee Gu\'edon, Val\'erie Hayot-Sasson, Tristan Glatard \\
  \IEEEauthorblockA{
    Department of Computer Science and Software Engineering, Concordia University, Montreal, Canada
  }}
}

\maketitle

\begin{abstract}
Todo.
\end{abstract}

\begin{IEEEkeywords}
multidimensional, array, split, merge, resplit, IO, processing, Dask, Python
\end{IEEEkeywords}

%----------------------------------------
\section{Introduction}
%----------------------------------------
% big data challenges
With the improvement of acquisition methods in several scientific domains such as
health sciences, geology and astrophysics, new big data challenges have emerged.
Such challenges includes processing big amounts of data and manipulating heavy
files like ultra high resolution images. Big Brain, a brain model providing
microscopic data (20 micrometers) for modeling and simulation~\cite{Amunts1472}
is an example of ultra high resolution images in the neuroscience field.

\subsection{Multidimensional array chunking}
% use of chunking
On the one hand, scientific applications often model problems as
multidimensional arrays that must be stored in chunks for later processing and analysis.
Chunking allows for efficient queries and great flexibility in adding new data, among other things.
On the other hand, block algorithms are necesary when data cannot be entirely
loaded in memory and to parallelize computations.

% need for tools to rechunk
Multidimensional array chunking raises the need for tools to efficiently split,
merge and ``resplit" or ``rechunk" data files. Previous work in ~\cite{seqalgorithms}
show that naive algorithms to split an array into several files and merge back those
files into one output file perform very poorly due to millions of seeks occuring on disk.

\subsection{The occurrence of seeks}
% explain storage order
For this study we manipulate multidimensional arrays stored in hdf5 files,
that is why we assume that files are written in row-major order (a.k.a ``C" ordering),
where the fastest moving dimension in the file is the last dimension in the
array and the slowest moving dimension in the file is the first dimension
in the array. For example a 3D array with dimensions i, j and k would be
written on disk by writing the complete columns in the k dimension first.
As in previous work~\cite{seqalgorithms}, we assume that the storage layout can
be arbitrary as long as it is known to the algorithm.

% why subarray extraction incurs seeks
Retrieving subarrays from an array stored on disk incurs seeks. When accessing a
multi-dimensional array, seeks occur in two situations:
(1) when an array block is opened for reading or writing, and (2) when the
reading or writing process moves within the block to write at different
places. While reading the data of interest, one needs to seek to the
beginning of each contiguous piece of data
constituting the subarray. For a 3D cuboid of shape C, stored in row-major order,
each column of data in the k dimension is contiguously stored. They are therefore
Cj * Ci data columns for which a seek is required on disk. Although the seek
time depends on various parameters such as the distance in bytes between two
data columns, we assume all seeks to incur the same time overhead for the sake
of simplicity. We further assume that reducing the number of seeks is representative
of reducing the processing speed.

\subsection{The multiple and clustered strategies}
Two strategies have been introduced in~\cite{seqalgorithms} for the split and
merge tasks. To illustrate these strategies we shall apply it to the split task
(R = I) as it is considered a dual operation of the merge task.
The naive strategy to split an array is to repetedly loading one
subarray of shape O and writing it down into an output file~\cite{seqalgorithms}.

The clustered strategy consists in loading contiguous blocks of data of shape O
from the input array in order to write it in one seek in each output file which
data has been loaded. This strategy loads data blocks one by one, block columns
by block columns or blocks slices by block slices. Using this strategy for
resplitting would imply reading multiple input files that are contiguous in the
reconstructed image or reading the data of contiguous output files by seeking into
input files. In either case one can easily find bad cases in which the algorithm
seeks in all dimensions.

The multiple strategy aims at not doing any seek while reading and writing. For
example, one would read the input file slice by slice and write this slices
contiguously in the output files. The tradeoff lies in switching between files
at each buffer loading. Using this strategy for the resplit task would imply
reading contiguous columns or slices of data from the input files and writing it
contiguously in the output files. This would result in even more switches
between input and output files and one can easily find worst cases when using
small I and O shapes for example where a smarter strategy would be appreciated.

To the best of our knowedge however, no algorithm has been proposed for the resplit task.

\subsection{Problem definition}
We focus on 3D arrays for simplicity. Consider a 3D array of shape $R =
(R_i, R_j, R_k)$, stored as input blocks of uniform shape $I =
(I_i, I_j, I_k)$. Our goal is to resplit the input blocks into output
blocks of uniform but different shape $O = (O_i, O_j, O_k)$.

A resplit algorithm (Algorithm \ref{algo:generalresplit}) takes as input a
list of input blocks \texttt{inBlocks} of shape $I$, a list of output
blocks \texttt{outBlocks} of shape $O$, the amount of available memory
\texttt{m}, and a list of buffers \texttt{buffersList} defining a partition
of array $R$. The algorithm successively loads the image regions defined by
the buffers, and writes them to the output blocks.

For simplicity, we require that all buffers in \texttt{buffersList} have
the same shape. Our problem is to find the partition \texttt{bufferList}
that minimizes the number of seeks done by Algorithm
\ref{algo:generalresplit} given $I$ and $O$, subject to the amount $m$
of available memory.

\begin{algorithm}[H]
  \caption{General resplit algorithm}
  \label{algo:generalresplit}
  \begin{algorithmic}
    \STATE \textbf{Inputs:} {inFiles, outFiles}
    \STATE $B \leftarrow I$
    \STATE initialize(cache)
    \FOR{buffer in getBuffers($B$)}
      \STATE $\textrm{bufferData} \leftarrow \textrm{read(inFiles, buffer)}$
      \STATE insert(bufferData, cache)
      \FOR{$i$, $data$ in cache}
        \STATE write(outFiles[$i$], $data$)
      \ENDFOR
    \ENDFOR

  \end{algorithmic}
\end{algorithm}

\subsection{Baseline solution}
A naive, baseline algorithm for the resplit task is to load one input file at a
time into memory and write it in the appropriate output files. We can compute
the amount of seeks produced by this algorithm with the algorithm described in
Algorithm \ref{badref}. As one can see from the algorithm, unless the dimensions
of the input and output
files matches a considerable amount of seeks will occur. Also, an overlap in the
k dimension is more costly than one in the j dimension. Finally, if the output
file shape is greater than the input file shape it also incurs the maximum amount
of seeks.

\subsection{Considerations on shape mismatch}
If Ix != Ox in a given dimension x, we call it a shape mismatch. As it has been
shown in the paragraph on the baseline algorithm, if Ix < Ox and we read one
file at a time, we are condamned to seek in the output files at write time. If,
however, Ix > Ox, then we get remainders. Finally, if we just read what we need
from the input files to write in output files contiguously, we seek while reading
and writing. A solution to reduce the number of seeks seems to be to read more
than Ox and to elaborate a strategy to keep the remainders in memory instead of
writing it down directly until we get enough data to write it with a reduced
amount of seeks. Such a strategy is presented in the section keep algorithm.

\subsection{Contributions}
This paper makes the following contributions:
\begin{itemize}
  \item We define the resplit problem
  \item We propose a first sequential algorithm to efficiently resplit multidimensional arrays
  \item We give a public implementation of such algorithm in the Python library dask
  \item We enable the use of the clustered strategy for splitting and merging multidimensional arrays in dask
\end{itemize}

%----------------------------------------
\section{The ``keep" algorithm}
%----------------------------------------

%----------------------------------------
\section{A memory analysis of the keep strategy}
%----------------------------------------
This section covers how we estimate the amount of memory required by the keep
strategy to know how much the buffer can be stretched in a given direction while
keeping the remainders in memory. In this analysis, we express the quantity of
memory used by an array as the number of voxels it contains. It is equivalent
to saying that the number of bytes per voxel is 1.

The amount of memory required is the maximum memory consumption reached during
the execution of the keep algorithm. To estimate this amount, we first need to
define what a remainder volume is. A buffer of shape $\Lambda$ can be divided in
8 parts or ``volumes". 7 out of those 8 volumes are remainder volumes because
they represent the overlap between the input aggregate and the output files on
its border. These 7 remainder volumes enclose an 8th part that is composed of
input files that are either complete or which complementary part have already
been loaded by a previous buffer. This means that any complementary part is either
in memory (it has been kept in cache according to the keep strategy) or it has
been written down previously. Therefore, this 8th volume cannot be kept by the
keep strategy and is not considered a remainder. For each buffer, each of volume
is indexed following the buffer order (see section on buffer order) G0 to G7, with
G0 being the non-remainder volume. If the buffer is smallest than the input
aggrgegate or if there is no shape mismatch in a given dimension, it may be that
a volume size is set to 0.

Having partitioned the buffers into such volumes, we can see that the maximum
memory consumption is computed from two pieces of information: The maximum number
of each volumes we must keep during the process and the maximum size of each
volume. The maximum number of each volumes that we must keep during the process
is (see computation details in Appendix~\ref{badref}):

\begin{equation} \label{eq:1}
\sigma = G_1 + n(G_2 + G_3) + N(G_4 + G_5 + G_6 + G_7)
\end{equation}

With $n$ the number of buffers in the first direction of the buffer order and
$N$ is the number of buffers in a buffer slice, i.e. in the second direction of
the buffer order. Given that at each step of the resplit algorithm one must
load a buffer, we should also add the size of a buffer to the equation:

\begin{equation} \label{eq:2}
\Sigma = (G_1 + n(G_2 + G_3) + N(G_4 + G_5 + G_6 + G_7) + B_iB_jB_k)
\end{equation}

Finally, to find the maximum size of each volume one must know the maximum
overlap length between the buffer and the output files on its border. One can
either compute all possibilities and take the maximum in each dimension if it is
not too costly or use the following upper bound: By definition of an input
aggregate, the overlap between any buffer and its bordering output files it at
maximum Ox in dimension x.

Given equation~\ref{badref} and the maximum size of the volumes, we can now
stretch the buffer shape one dimension at a time using simple equations to ensure
that the maximum amount of memory consumed during the process wil stay below the
amount of memory available. Such computations and the associated buffer stretching
algorithm are given in Appendix~\ref{badref}.

%----------------------------------------
\section{Methods}
%----------------------------------------

%----------------------------------------
\section{Implementation details}
%----------------------------------------

%----------------------------------------
\section{Results}
%----------------------------------------

%----------------------------------------
\section{Discussion}
%----------------------------------------

Future work :
- the split and merge tasks
are special cases of the resplit task, which gives an opportunity to solve
the three problems at once.

%----------------------------------------
\section{Conclusion}
%----------------------------------------

%----------------------------------------
\section{Acknowledgments}
%----------------------------------------

\bibliography{Bibliography}
\bibliographystyle{ieeetr}

\end{document}
